% Options for packages loaded elsewhere
\PassOptionsToPackage{unicode}{hyperref}
\PassOptionsToPackage{hyphens}{url}
%
\documentclass[
]{article}
\usepackage{amsmath,amssymb}
\usepackage{lmodern}
\usepackage{iftex}
\ifPDFTeX
  \usepackage[T1]{fontenc}
  \usepackage[utf8]{inputenc}
  \usepackage{textcomp} % provide euro and other symbols
\else % if luatex or xetex
  \usepackage{unicode-math}
  \defaultfontfeatures{Scale=MatchLowercase}
  \defaultfontfeatures[\rmfamily]{Ligatures=TeX,Scale=1}
\fi
% Use upquote if available, for straight quotes in verbatim environments
\IfFileExists{upquote.sty}{\usepackage{upquote}}{}
\IfFileExists{microtype.sty}{% use microtype if available
  \usepackage[]{microtype}
  \UseMicrotypeSet[protrusion]{basicmath} % disable protrusion for tt fonts
}{}
\makeatletter
\@ifundefined{KOMAClassName}{% if non-KOMA class
  \IfFileExists{parskip.sty}{%
    \usepackage{parskip}
  }{% else
    \setlength{\parindent}{0pt}
    \setlength{\parskip}{6pt plus 2pt minus 1pt}}
}{% if KOMA class
  \KOMAoptions{parskip=half}}
\makeatother
\usepackage{xcolor}
\IfFileExists{xurl.sty}{\usepackage{xurl}}{} % add URL line breaks if available
\IfFileExists{bookmark.sty}{\usepackage{bookmark}}{\usepackage{hyperref}}
\hypersetup{
  pdftitle={Edge/Centre Research Program},
  hidelinks,
  pdfcreator={LaTeX via pandoc}}
\urlstyle{same} % disable monospaced font for URLs
\usepackage[margin=1in]{geometry}
\usepackage{graphicx}
\makeatletter
\def\maxwidth{\ifdim\Gin@nat@width>\linewidth\linewidth\else\Gin@nat@width\fi}
\def\maxheight{\ifdim\Gin@nat@height>\textheight\textheight\else\Gin@nat@height\fi}
\makeatother
% Scale images if necessary, so that they will not overflow the page
% margins by default, and it is still possible to overwrite the defaults
% using explicit options in \includegraphics[width, height, ...]{}
\setkeys{Gin}{width=\maxwidth,height=\maxheight,keepaspectratio}
% Set default figure placement to htbp
\makeatletter
\def\fps@figure{htbp}
\makeatother
\setlength{\emergencystretch}{3em} % prevent overfull lines
\providecommand{\tightlist}{%
  \setlength{\itemsep}{0pt}\setlength{\parskip}{0pt}}
\setcounter{secnumdepth}{-\maxdimen} % remove section numbering
\ifLuaTeX
  \usepackage{selnolig}  % disable illegal ligatures
\fi

\title{Edge/Centre Research Program}
\author{}
\date{\vspace{-2.5em}}

\begin{document}
\maketitle

\hypertarget{education---human-services---arts---imagination---research---practice--}{%
\paragraph{- Education - Human Services - Arts - Imagination - Research
- Practice
-}\label{education---human-services---arts---imagination---research---practice--}}

\includegraphics{https://raw.githubusercontent.com/michael-emslie/Edge-Centre.github.io/master/Screenshot\%20(10).png}

\begin{center}\rule{0.5\linewidth}{0.5pt}\end{center}

\hypertarget{introducing-the-edgecentre-research-program}{%
\subsubsection{Introducing the Edge/Centre Research
Program}\label{introducing-the-edgecentre-research-program}}

The Edge/Centre includes over 25-years of research and social action
collaborations that have transformed into the productive research
program that we now call the Edge/Centre. Aspects of the history of this
work are documented and includes hosting conferences, community
development activities, facilitation of workshops, lobbying, delivery of
art-based sessions, participation on steering committees, activism,
conference presentations, providing professional development, writing
submissions, research projects, and writing, e.g., Crowhurst and Emslie,
2020, 2018, 2014, 2003, 2000a, 2000b.

The Edge/Centre is a space of collaborative scholarly endeavours and
creatively designed intellectual pursuits. The Edge/Centre maintains a
key focus on investigating and deploying creative research methods and
innovative knowledge practices to generate and enhance the possibilities
for imaginative ethical professional practice in human services,
education and the arts.

\begin{center}\rule{0.5\linewidth}{0.5pt}\end{center}

\hypertarget{what-do-we-mean-by-the-edgecentre}{%
\subsubsection{What do we mean by the
Edge/Centre?}\label{what-do-we-mean-by-the-edgecentre}}

In Crowhurst and Emslie (2018, p.~ix) we describe the Edge/Centre as;

\begin{quote}
`\ldots an idea and a location where the values of collaboration,
emergence, dialogue and creativity drive the generation
of\ldots{[}the{]} text.'
\end{quote}

The Edge/Centre is an idea and a space we inhabit intellectually,
emotionally, affectively and physically to think, research, read, play,
dialogue, reflect, diffract, squiggle, collaborate, compost, analyse,
and write.

The Edge/Centre involves intentionally locating intellectual, emotional
and physical labour on the margins. For example, we situate our research
and writing as being on the edge of what is typical, prevailing,
mainstream, aligned, assumed and expected. Our work is deliberately done
on the margins, not in the centre of things, and does not seek to follow
mainstream agendas because we find many aspects of these problematic.

\begin{quote}
'We think we write from an edge, an edge that is quirky and different
and where our marginality allows for, generates and affords other
interesting kinds of connection\ldots{[}One thing our work{]} is about
storying edges and the generative possibilities that marginality can
offer (Crowhurst and Emslie, 2020, p.~6).
\end{quote}

The Edge/Centre is an alternative space, a place of difference, that is
`out there', edgy, that involves doing edginess, pushing boundaries, and
doing things in unexpected and not what is expected or assumed ways. The
Edge/Centre also functions as a form of support where edgy work is
validated and celebrated.

The Edge/Centre is a `pleasurable non-normative space' that we claim,
occupy, inhabit, are captured by, are drawn to and purposefully build as
an alternative to the existing middle or neoliberal centre (Crowhurst
and Emslie, 2020, p.~23). An example of the effects generated on account
of inhabiting the Edge/Centre is the form that our writng takes.

\begin{quote}
`This edginess is reflected in the texts' style and in its eclecticism,
in its tone, and in its voice and register' (Crowhurst and Emslie, 2020,
p.~6).
\end{quote}

\begin{center}\rule{0.5\linewidth}{0.5pt}\end{center}

\hypertarget{storying-the-edgecentres-recent-research-projects-and-future-initiatives}{%
\subsubsection{\texorpdfstring{Storying the Edge/Centre's recent
research projects and future initiatives
}{Storying the Edge/Centre's recent research projects and future initiatives  }}\label{storying-the-edgecentres-recent-research-projects-and-future-initiatives}}

\href{https://www.palgrave.com/gp/book/9783319697536}{\textbf{`Working
Creatively with Stories and Learning Experiences: Engaging with Queerly
Identifying Tertiary Students'}}

\includegraphics{https://raw.githubusercontent.com/michael-emslie/Edge-Centre.github.io/master/Screenshot\%20(6).png}

Our first book (Crowhurst and Emslie, 2018) outlines a series of methods
to work creatively with stories, to open up possibilities in
understanding and thinking about stories, and to open up possibilities
in stories for generating understanding and meaning in our lives.
`Working Creatively with Stories and Learning Experiences: Engaging with
Queerly Identifying Tertiary Students' explores how we can work
creatively with stories and we used the stories of queerly identifying
tertiary students to demonstrate this. In the book we describe and
deploy a range of methods to analyse the stories of the students. We
make the case that the stories, and the subjectivities they capture are:
* (ch 2) discursively produced (or an effect of discourse in context) *
(ch 3) performative (or performed, enacted and embodied) * (ch 4)
multiple (or consisting of an assemblage of elements) Then (ch 5) we
argue that the elements of these discursively produced, performative and
multiple/fragmented stories/subjectivities can at times be contrary and
sit in tension. We argue there is a tendency to ignore, overlook, deny,
minimise, and erase such tensions. However, we make the case that such
tension, ambiguity, diversity, and complexity in stories/subjectivities
has productive potential and is generative of other possibilities. In
(ch 6) we make the case that `reading aloud like a playscript' is a way
to engage with these possibilities. We argue that these methods can be
used to analyse the generative potential of any story whether they be
about people, practice, policy, institutions, or contexts.

\begin{quote}
\emph{Praise for `Working Creatively with Stories and Learning
Experiences: Engaging with Queerly Identifying Tertiary Students':}
\end{quote}

\emph{`At a time when gender and sexual diversity are high on the
agenda, this book marks a timely intervention in the tertiary education
context. It will be of interest and value to a wide range of
professionals working to support LGBTQ students in colleges and
universities.'} \emph{- Prof Peter Aggleton, University of New South
Wales, Australia}

\emph{`This is a very useful guide not only in support of LGBTIQ issues
but also a very accessible academic overview of using, analyzing and
creatively engaging with narratives in many disciplines, with many
communities, and in multiple settings.'} \emph{- Dr Maria Pallotta
Chiarolli, Deakin University, Australia}

\href{https://www.springer.com/gp/book/9783030375065}{\textbf{`Arts-based
Pathways into Thinking: Troubling Standardization/s, Enticing
Multiplicities, Inhabiting Creative Imaginings'}}

\includegraphics{https://raw.githubusercontent.com/michael-emslie/Edge-Centre.github.io/master/Screenshot\%20(8).png}

Our second book (Crowhurst and Emslie, 2020) turns to our stories and we
deploy a critical/collective/auto/ethnographic approach to engage with
these accounts. Then using the methods from our first book and
introducing a series of arts-based research methods we work with our
stories to generate possibilities for understanding and thinking about
stories in different ways. `Arts-based Pathways into Thinking: Troubling
Standardization/s, Enticing Multiplicities, Inhabiting Creative
Imaginings' extends the ideas from our first book. In this book (ch 1)
we argue that the beginning of the 21st century has seen a marked
interest in creative research methods and novel knowledge practices. At
the same time, we make the case (ch 2) that we are living in
increasingly standardized times that feature systems that specify
objectives ahead of time, demand compliance, and narrow the
possibilities for human action. It is this contradictory set of
conditions that incited and generated this book. In the book we describe
and deploy an assemblage of theoretically informed, creative, art-based,
and creative/critical/collective/auto/ethnographic methods that aim to
trouble and resist standardizations and normativities and provoke
multiplicities and the inhabitation of creative imaginings. We provide
accounts and analysis of the incitement and inhabitation of
multiplicities of knowing, sensing and doing generated by analysing
knowing frames (ch 3), poetry (ch 4), reading aloud (ch 5), combinings
(ch 6), fable-ing (ch 7), and other creative ways of working with
experiences and stories. We argue that the generation and inhabitation
of these multiplicities sheds light on the cultural conditions that
support and enable diversity and creativity. The book takes readers on
an arts-based journey designed to enhance the opportunities for
imaginative and ethical professional practice in education, human
services and the arts.

\begin{quote}
\emph{Praise for `Arts-based Pathways into Thinking: Troubling
Standardization/s, Enticing Multiplicities, Inhabiting Creative
Imaginings':}
\end{quote}

\emph{`Offering creative and nuanced arts-based pathways into thinking
and practices, this pioneering book engages us in a variety of ways and
at all levels\ldots This is simply stunning. Stunning to read. Stunning
to view\ldots{}'} \emph{- Prof Pam Barnard, University of Cambridge,
Cambridge, UK}

\emph{`Crowhurst and Emslie offer a compelling and scholarly model of
collective, arts-based autoethnography in action. They challenge the
reader to interrogate relationships with students and colleagues,
deconstructing the forces of standardization to reveal a multiplicity of
understandings and narratives.'} \emph{- Dr Andrea Kantrowitz, State
University of New York, USA}

\href{https://link.springer.com/book/10.1007/978-981-16-9400-4}{\textbf{On
Pedagogical Spaces, Multiplicty and Linearities and Learning: Before,
Between, Beyond}}

\includegraphics{https://raw.githubusercontent.com/edge-centre/edge-centre.github.io/master/Screenshot\%20(15).png}

\textbf{`Work in progress\ldots{}'}

Our next book (which is in progress), continues to explore and describe
more inventive, playful, scholarly and theoretically informed methods
for working creatively with stories, generating possibilities, and
thinking stories in different ways. At this point this new book focuses
on different modalities of thinking - the analytical, the attuned, the
diffractive and the imaginative.

In `Philosophy and the Event' Badiou suggests that change doesn't happen
in monolithic ways but rather it happens when people realize where they
are, decide that this place is problematic, think into a somewhere else
that is better and then decide to take small steps towards or into this
better somewhere else. The Edge/Centre is about making such small steps
and about somewhere else-ing.

\begin{center}\rule{0.5\linewidth}{0.5pt}\end{center}

\hypertarget{in-the-media}{%
\subsubsection{\texorpdfstring{In the media
}{In the media  }}\label{in-the-media}}

\href{https://mms.tveyes.com/mediaview/?U3RhdGlvbj04Njc1JlN0YXJ0RGF0ZVRpbWU9MTAlMkYxMiUyRjIwMjAlMjAwNSUzQTM2JTNBNDUmRW5kRGF0ZVRpbWU9MTAlMkYxMiUyRjIwMjAlMjAwNSUzQTQ2JTNBNDUmUGxheVN0YXJ0UmVnZXg9JTVDYkRyJTVDYiU3QyU1Q2JSTUlUJTVDYiZQbGF5U3RhcnRSZWdleFByZXJvbGw9MTUmRHVyYXRpb249Mjk4ODkwJlBhcnRuZXJJRD03MzEzJkV4cGlyYXRpb249MTAlMkYxMyUyRjIwMjAlMjAwNiUzQTIxJTNBNDImSGlnaGxpZ2h0UmVnZXg9JTVDYkRyJTVDYiU3QyU1Q2JSTUlUJTVDYiZNb2RFZGl0b3JFbmFibGU9dHJ1ZSZNb2RFZGl0b3JEZXN0aW5hdGlvbnM9NCZzaWduYXR1cmU9YWVlNjhjYzFhNzlkNDZlMGExMjRmYzMxMjU1NzQ2Njc=\#}{720
ABC Perth 12/10/2020}

\begin{center}\rule{0.5\linewidth}{0.5pt}\end{center}

\hypertarget{look-at-me-christmas-a-poetic-critique-on-the-use-of-shiny-surfaces-in-education}{%
\subsubsection{Look at me Christmas! A poetic critique on the use of
shiny surfaces in
education}\label{look-at-me-christmas-a-poetic-critique-on-the-use-of-shiny-surfaces-in-education}}

In December each year The children would line up Toffee apples and
smiles Gifts for miles and miles

Chorus: So much Christmas! Shiny, shiny Christmas! Look at me Christmas!
Christmas, we become

All anticipating Christmas It couldn't come fast enough Only 25 sleeps
to go Oh Rudolph, elves, sleighs and snow

(Chorus)

The Christmas windows sparkled All green and white and red And Santa
loved the way they drew attention to his furry furry head

(Chorus)

\begin{center}\rule{0.5\linewidth}{0.5pt}\end{center}

\hypertarget{further-information-contact-us}{%
\subsubsection{Further information / Contact
us:}\label{further-information-contact-us}}

Have a question? Want to learn more? Interested in getting involved?
Like us to deliver a professional development workshop or research
seminar? Please contact us:

\begin{quote}
\textbf{Dr Michael Crowhurst}
\href{mailto:michael.crowhurst@rmit.edu.au}{\nolinkurl{michael.crowhurst@rmit.edu.au}}
\end{quote}

\begin{quote}
\textbf{Dr Michael Emslie}
\href{mailto:michael.emslie@rmit.edu.au}{\nolinkurl{michael.emslie@rmit.edu.au}}
\end{quote}

\begin{center}\rule{0.5\linewidth}{0.5pt}\end{center}

\hypertarget{references}{%
\paragraph{\texorpdfstring{References
}{References  }}\label{references}}

\href{https://www.springer.com/gp/book/9783030375065}{Crowhurst, M. \&
Emslie, M. 2020. Arts-Based Pathways into Thinking: Troubling
Standardization/s, Enticing Multiplicities, Inhabiting Creative
Imaginings, Springer, Cham, Switzerland.}

\href{https://www.palgrave.com/gp/book/9783319697536}{Crowhurst, M.\&
Emslie, M. 2018. Working creatively with stories and learning
experiences: Engaging with queerly identifying tertiary students.
Springer, Cham, Switzerland.}

\href{https://www.tandfonline.com/doi/abs/10.1080/19361653.2013.879466}{Crowhurst,
M. \& Emslie, M. 2014 `Counting queers on campus: Collecting data on
queerly identifying students', Journal of LGBT Youth, vol.~11, no. 3,
pp.~276-288.}

\href{https://www.researchgate.net/publication/262451596_Doing_new_work_Materials_for_queer_teachers_and_youth_workers}{Crowhurst,
M. \& Emslie, M. (eds) 2003 Doing new work: Materials for queer teachers
and youth workers, Youth Research Centre, University of Melbourne,
Melbourne.}

\href{https://www.researchgate.net/publication/262451598_Young_People_and_Sexualities_Experiences_Perspectives_and_Service_Provision_Papers_from_a_Community_Conference_exploring_issues_affecting_gay_lesbian_bisexual_and_transgender_young_people_October_1998}{Crowhurst,
M. \& Emslie, M. (eds) 2000a Young People and Sexualities: Experiences,
Perspectives and Service Provision: Papers from a Community Conference
exploring issues affecting gay, lesbian, bisexual and transgender young
people; October 1998, Youth Research Centre, University of Melbourne,
Melbourne.}

\href{https://www.researchgate.net/publication/317099886_Context_A_Model_for_Action_Around_Issues_Affecting_gay_Lesbian_Bisexual_and_Transgender_Teachers_and_Student}{Crowhurst,
M. \& Emslie, M. 2000b. Context: A Model for Action Around Issues
Affecting gay, Lesbian, Bisexual and Transgender Teachers and Students,
delivered to the Australian Association for Research in Education (AARE)
Conference `Education Research: Towards an Optimistic Future', Sydney
2000.}

\begin{center}\rule{0.5\linewidth}{0.5pt}\end{center}

\end{document}
